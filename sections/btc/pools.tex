\subsection{Bitcoin Pool Contribution}\label{sec:btc-pools}
\subsubsection{Assumptions}
\begin{enumerate}
  \item The operating hours of a Bitcoin pool is 24 hours per day, 7 days per week, 52 weeks per year, for a total of 8,760 hours per year.
  \item Pools do not typically report data regarding their infrastructure, likely because this is not commonly requested information.
  Upon reaching out to Slushpool, a prominent Bitcoin mining pool, they readily provided the Galaxy Digital Mining Team with information on their infrastructure requirements.
  Slushpool's infrastructure is used to estimate their total annual electricity usage based on the demand of a typical server.
  To estimated the total annual electricity usage of all Bitcoin mining pools, a linear relationship between a pool's server demand and the number of mined blocks in a 4 day period is assumed.
\item The office space of the Bitcoin mining pools is not considered, but is a negligible amount as there are likely only a few offices globally.
\item 153 blocks mined in the considered 4-day period do not come from known pools, but are instead from unknown miners.
  In order to be conservative, these blocks are still included in this analysis because they still may be mined via a pool, not necessarily an individual miner.
\end{enumerate}

\subsubsection{Methodology}
The estimated total annual electricity usage required to operate Bitcoin mining pools is calculated as follows:

\begin{align}\label{eqn:btc-pool-e}
  E_\textrm{Pools} = blk_\textrm{T} \times \frac{D_\textrm{SP}}{blk_\textrm{SP}} \times H_\textrm{yr} \times C
\end{align}

\noindent
Where,

\begin{align*}
&E_\textrm{Pools} \textrm{ = Annual eletricity usage of Bitcoin pools, $\xwhyr{T}$} \\
&blk_\textrm{T}   \textrm{ \ \ = Number of blocks mined in a 4 day period, 594 blocks \cite{mined-blocks}} \\
&blk_\textrm{SP}  \textrm{ \ = Number of blocks mined by Slushpool in a 4 day period, 19 blocks \cite{mined-blocks}} \\
&D_\textrm{SP}    \textrm{ \quad = Demand of Slushpool's servers, 19.5 kW} \\
&C                \textrm{\qquad \ = Conversion factor, $\convtk$}
\end{align*}

The operating hours of a Bitcoin pool is 24 hours per day, 7 days per week, 52 weeks per year, for a total of 8,760 hours per year.
The infrastructure requirements to operate a pool are minimal and are not typically reported by a pool, likely due to lack of public interest and because the requirements are minimal.
However, the Galaxy Digital Mining team reached out to Slushpool who provided details of their server infrastructure requirements, which is detailed in Table \ref{table:slushpool-infra}.
A demand for a typical 1U server of 300 W is assumed \cite{1u-power} and is used to estimate Slushpool's server demand.
In order to estimate the total annual electricity usage of all Bitcoin mining pools, a linear relationship between a pool's server demand and the number of mined blocks in a four day period is assumed.
The four day period is from April 19, 2021 through April 22, 2021.
During this period, 594 blocks were mined in total, 19 of which were mined by Slushpool \cite{mined-blocks}.
The demand of Slushpool's servers is detailed in Section \ref{sec:btc-slushpool-server} below.
Eq. \eqref{eqn:btc-slush-e-res} in Section \ref{sec:btc-slushpool-server} below finds Slushpool's server demand to be 19.50 kW.

Eq. \eqref{eqn:btc-pool-e} is solved as follows:

\begin{align}\label{eqn:btc-pool-e-rew}
  E_\textrm{Pools} &= B_\textrm{Total} \times \frac{D_\textrm{SP}}{B_\textrm{SP}} \times H_\textrm{yr} \times C \\
                   &= 594 \textrm{ blocks}\times \frac{19.50 \textrm{ kW}}{19 \textrm{ blocks}} \times 8,760 \hyr \times \convtk \nonumber \\
                   &= 8,609,328 \kwhyr \times \convtk \nonumber \\
                   &= 0.0086 \xwhyr{T} 
\end{align}

Therefore, the estimated total annual electricity usage required to operate the Bitcoin mining pools is 0.0086 $\xwhyr{T}$.
